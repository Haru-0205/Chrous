\documentclass{jlreq}
\usepackage{hyperref}
\begin{document}
\begin{flushright}
    令和5年8月7日
\end{flushright}
\begin{flushleft}
    公大高専1年各位
\end{flushleft}
\begin{flushright}
    大阪公立大学工業高等専門学校\\
    総合工学システム学科1年4組副総代\\
    谷口 陽音\\
    \href{mailto:rm23081b@osaka-pct.ac.jp}{\nolinkurl{rm23081b@osaka-pct.ac.jp}}\\
\end{flushright}
\begin{center}
    \begin{Large}
        合唱プロジェクトについて
    \end{Large}
\end{center}
拝啓 時下ますますご清祥のこととお慶び申し上げます。\\
さて、この度、私を中心としまして、合唱プロジェクトを立ち上げることになりました。\\
このプロジェクトは、私が音楽科の木村先生に合唱について聞いたところから始まりました。\\
音楽科のほうでは時間的に厳しいため、有志の方々を集めて、合唱プロジェクトを立ち上げることにしました。\\
参加を希望される方や、迷われる方は下記のイベントにご参加ください。\\
なお、イベントに参加する前から合唱のプロジェクトに携わることは可能です。\\
その場合は下記連絡先までご連絡ください。\\
\begin{flushright}
    敬具
\end{flushright}
\begin{center}
    記\\
\end{center}
\begin{enumerate}
    \item イベント名\\
          合唱プロジェクト説明会
    \item イベント日時\\
          令和5年8月10日(木) 20:00-21:30
    \item イベント場所\\
          オンライン開催\\
          \href{https://youtube.com/live/JUqEJS81dz4?feature=share}{YouTube Live}\\
    \item イベント内容\\
          プロジェクトの説明\\
          今後の動きについて\\
          参加方法について
    \item 連絡先\\
          代表:谷口陽音\\
          \href{mailto:haru-0205@proton.me}{\nolinkurl{haru-0205@proton.me}}\\
          \href{mailto:haru-0205@outlook.com}{\nolinkurl{haru-0205@outlook.com}}\\
          \href{mailto:rm23081b@osaka-pct.ac.jp}{\nolinkurl{rm23081b@osaka-pct.ac.jp}}\\
          \href{sms:080-1425-9087}{\nolinkurl{080-1425-9087}}\\
    \item 注意事項\\
          YouTube Liveの限定公開アドレスを共有しないでください。\\
          YouTubeにおける限定公開コンテンツを含むパブリック再生リストは作成しないでください。\\

\end{enumerate}
\begin{flushright}
    以上
\end{flushright}
\end{document}